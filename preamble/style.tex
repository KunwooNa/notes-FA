% ------ Basic packages -------
\usepackage{sectsty}
\chapternumberfont{\Large} 
\chaptertitlefont{\huge}
\usepackage{amsmath}
\usepackage{amssymb}
 \usepackage{mathtools}
 \usepackage[amsthm, thmmarks,amsmath]{ntheorem}
\usepackage{geometry}
\geometry{left=3cm,right=3cm,top=3.4cm,bottom=3.4cm}
\usepackage{kotex} 
\usepackage{stmaryrd}
\usepackage[hyphens,spaces,obeyspaces]{url}
\usepackage{algorithm}
\usepackage{algpseudocode}
% \usepackage[sectionbib]{chapterbib}
\usepackage[sectionbib]{natbib}
\usepackage{chapterbib}
\usepackage[nottoc]{tocbibind}
\usepackage{lipsum}
\usepackage{fancyhdr,lastpage}
\usepackage{paralist}
\usepackage[usenames,svgnames,dvipsnames]{xcolor}
\usepackage{color}
\setcounter{chapter}{1}
\usepackage{authblk}



% ------ Version Controler ------
\usepackage{version}
\includeversion{solution}






% ----- Listings Related Things -----
\usepackage{listings}
\usepackage{xcolor}
\lstset { %
    %language=python,
    backgroundcolor=\color{black!5}, % set backgroundcolor
    %basicstyle=\ttfamily,
    basicstyle= \sffamily,
    %commentstyle=\ttfamily,
    commentstyle = \sffamily, 
    showstringspaces=false
}

\usepackage{tikz}
\usepackage{pgfplots}
\pgfplotsset{compat=1.17}
\usetikzlibrary{patterns}
\makeatletter
\newcommand{\pgfplotsdrawaxis}{\pgfplots@draw@axis}
\makeatother
\pgfplotsset{only axis on top/.style={axis on top=false, after end axis/.code={
             \pgfplotsset{axis line style=opaque, ticklabel style=opaque, tick style=opaque,
             grid=none
             }\pgfplotsdrawaxis}
}}
\newcommand{\drawge}{-- (rel axis cs:1,0) -- (rel axis cs:1,1) -- (rel axis cs:0,1) \closedcycle}
\newcommand{\drawle}{-- (rel axis cs:1,1) -- (rel axis cs:1,0) -- (rel axis cs:0,0) \closedcycle}


\setlength{\parindent}{0in}



% ----- Styles ----- 
\usepackage{tikz-cd}
\usetikzlibrary{arrows,arrows.meta}
% make a larger hook
% https://tex.stackexchange.com/questions/514451/how-to-define-a-new-hooked-arrow
\makeatletter
\pgfdeclarearrow{
  name=xGlyph,
  cache=false,
  bending mode=none,
  parameters={\tikzcd@glyph@len,\tikzcd@glyph@shorten},
  setup code={%
    \pgfarrowssettipend{\tikzcd@glyph@len\advance\pgf@x by\tikzcd@glyph@shorten}},
  defaults={
    glyph axis=axis_height,
    glyph length=+1.55ex,
    glyph shorten=+-0.1ex},
  drawing code={%
    \pgfpathrectangle{\pgfpoint{+0pt}{+-1.5ex}}{\pgfpoint{+\tikzcd@glyph@len}{+3ex}}%
    \pgfusepathqclip%
    \pgftransformxshift{+\tikzcd@glyph@len}%
    \pgftransformyshift{+-\tikzcd@glyph@axis}%
    \pgftext[right,base]{\tikzcd@glyph}}}
\makeatother
\tikzcdset{
	arrow style=tikz,
	diagrams={>={Latex}},
	tikzcd left hook/.tip={xGlyph[glyph math command=supset, swap, glyph axis = 5.7pt]},
	tikzcd right hook/.tip={xGlyph[glyph math command=supset, glyph axis = 5.7pt]}
}




% ----- Theorems ------- 
\usepackage{thmtools}
\usepackage[framemethod = TikZ]{mdframed}



% ---- Problem counter ----
\newcounter{pbctr}[chapter]
\setcounter{pbctr}{0}



\mdfdefinestyle{mdblackbox}{%
	skipabove=8pt,
	linewidth=3pt,
	rightline=false,
	leftline=true,
	topline=false,
	bottomline=false,
	linecolor=black,
	backgroundcolor=RedViolet!5!gray!5,
}
\declaretheoremstyle[
	headfont=\bfseries,
	bodyfont=\normalfont\small,
	spaceabove=0pt,
	spacebelow=0pt,
	mdframed={style=mdblackbox}
]{thmblackbox}


\declaretheorem[%
style=thmblackbox, name=Theorem, numberwithin=section]{theorem}
\declaretheorem[%
style=thmblackbox, name=Lemma, sibling = theorem]{lemma}
\declaretheorem[%
style=thmblackbox, name=Corollary, sibling = theorem]{corollary}
\declaretheorem[%
style=thmblackbox, name=Proposition, sibling = theorem]{proposition}
\declaretheorem[%
style=thmblackbox, name=Example, sibling = theorem]{example}



\mdfdefinestyle{mdgreenbox}{%
	skipabove=8pt,
	linewidth=3pt,
	rightline=false,
	leftline=true,
	topline=false,
	bottomline=false,
	linecolor=black,
	backgroundcolor=ForestGreen!3!
}
\declaretheoremstyle[
	headfont=\bfseries,
	bodyfont=\normalfont\small,
	spaceabove=0pt,
	spacebelow=0pt,
	mdframed={style=mdgreenbox}
]{exercisegreenbox}


\declaretheorem[%
style=exercisegreenbox, name=Exercise, numberwithin=chapter]{exercise}
